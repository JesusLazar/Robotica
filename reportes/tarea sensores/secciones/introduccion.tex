\section{Introducción}

Los sensores son componentes críticos en los sistemas robóticos, ya que permiten a los robots percibir y responder a su entorno. Estos dispositivos se clasifican en dos categorías principales: \textbf{sensores internos} y \textbf{sensores externos}. Los sensores internos se utilizan para monitorear el estado interno del robot, como su posición, velocidad, aceleración y fuerzas aplicadas. Por otro lado, los sensores externos permiten al robot interactuar con su entorno, detectando objetos, obstáculos y otras características relevantes \cite{UAEMex}.

En robótica, los sensores internos son esenciales para el control y la navegación, ya que proporcionan información en tiempo real sobre el estado del robot. Esta información es utilizada por el controlador para tomar decisiones y generar comandos de control. Por ejemplo, los encoders y potenciómetros son ampliamente utilizados para medir la posición, mientras que los tacómetros y sensores de efecto Hall se emplean para medir la velocidad \cite{OMCH, EfectoHall}.

En esta sección, se describen los principales tipos de sensores internos utilizados en robótica, incluyendo sensores de posición, velocidad, aceleración y fuerza. Cada uno de estos sensores desempeña un papel crucial en el control y la navegación de los robots, permitiendo que estos sistemas operen de manera precisa y eficiente.


\section{Sensores}
Un sensor es un dispositivo que detecta el cambio en el entorno y responde a alguna salida en el otro sistema. Un sensor convierte un fenómeno físico en un voltaje analógico medible (o, a veces, una señal digital) convertido en una pantalla legible para humanos o transmitida para lectura o procesamiento adicional. 
