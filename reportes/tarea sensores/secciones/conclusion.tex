\section{Conclusión} \label{sec:conclusion}

En esta investigación se analizaron los sensores internos y externos, destacando sus características, aplicaciones y diferencias clave. Los sensores internos, como los encoders, potenciómetros, LVDT y acelerómetros, son fundamentales para monitorear variables dentro de un sistema, como la posición, velocidad, aceleración y fuerzas aplicadas. Estos sensores son esenciales en aplicaciones como la robótica, la automatización industrial y los sistemas de control de vehículos \cite{UAEMex, Acelerometro}.

Por otro lado, los sensores externos, como los de proximidad, ultrasonidos, LiDAR y visión, permiten la interacción con el entorno, midiendo factores como la distancia, la presencia de objetos y las características del ambiente. Estos sensores son ampliamente utilizados en aplicaciones de automatización, domótica, seguridad y vehículos autónomos \cite{Proximidad, Ultrasonicos, LIDAR}.

El desarrollo de ambas categorías de sensores ha sido impulsado por los avances tecnológicos, especialmente en áreas como la miniaturización, la inteligencia artificial y la tecnología MEMS (Micro-Electro-Mechanical Systems). Estos avances han mejorado la precisión, eficiencia y accesibilidad de los sensores, permitiendo su integración en una amplia gama de aplicaciones \cite{MEMS}.

La elección entre sensores internos o externos depende de la aplicación específica y de los requisitos de medición. Por ejemplo, en un robot industrial, los sensores internos son cruciales para controlar el movimiento y la fuerza de los actuadores, mientras que los sensores externos son necesarios para detectar obstáculos y garantizar la seguridad en el entorno de trabajo \cite{Resolver, Proximidad}.

En conclusión, la integración de sensores internos y externos seguirá evolucionando, optimizando procesos y permitiendo sistemas más inteligentes y autónomos en diversas industrias. La combinación de estas tecnologías, junto con el uso de algoritmos avanzados de procesamiento de datos, abrirá nuevas posibilidades en campos como la robótica autónoma, la industria 4.0 y la Internet de las cosas (IoT) \cite{LIDAR, Vision}.




