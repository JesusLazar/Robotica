\section{Arvizu Michelle}
Durante el desarrollo de este proyecto, enfrenté varios retos que, aunque frustrantes en su momento, terminaron siendo grandes oportunidades de aprendizaje. Uno de los principales problemas fue entender y aplicar correctamente los conceptos de cinemática inversa. Al principio, me costó mucho lograr que el algoritmo convergiera a una solución estable; algunas veces el robot no alcanzaba la posición deseada o se movía de forma errática. Me tomó tiempo entender cómo ajustar los parámetros del algoritmo, como la tolerancia y el paso, y cómo evitar errores por cercanía a singularidades.

Otro problema que tuve fue con la simulación en Ubuntu. No tenía experiencia previa con herramientas como Gazebo, RViz o MoveIt, y tampoco con el uso de una máquina virtual. Instalar todo correctamente, hacer que el robot se visualizara y que respondiera a los comandos fue un proceso largo, con muchos errores de dependencia y configuración. Pero gracias al tutorial, a la documentación del repositorio y a la ayuda del equipo, logré completar la simulación y ver el robot moverse con sus dos pinzas coordinadas, lo cual me dio mucha satisfacción.