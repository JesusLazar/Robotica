\section{Montoya Morelia}
Me enfoqué principalmente en la parte de control, aunque al principio no entendía muy bien cómo aplicarlo en este contexto. Fue difícil interpretar cómo el Jacobiano influye en la estabilidad del robot y cómo se puede usar para diseñar leyes de control. No logramos implementar un control avanzado como queríamos, pero el simple hecho de entender cómo se conectan las velocidades articulares con la posición del efector fue muy valioso para mí.

Además, fue la primera vez que trabajé con ROS y todas las herramientas que conlleva: instalar Gazebo, configurar URDF, y hacer que todo corriera en WSL fue bastante abrumador. A pesar de eso, ahora me siento mucho más cómodo con estas herramientas y sé que podré usarlas con más soltura en futuros proyectos. Lo más importante que me llevo es que la robótica no es solo matemática y código: también es prueba, error, intuición y trabajo en equipo.

