\section{Carranza Jesús}
Lo que más aprendí fue a no rendirme cuando las cosas no salen a la primera. A veces pasaba horas intentando que algo funcionara y parecía imposible, pero con paciencia y pruebas fui entendiendo cómo se conectaban las piezas. También aprendí mucho sobre trabajo en equipo, comunicación y organización del código. Aunque al principio descargamos el modelo del robot desde internet, adaptarlo a nuestro proyecto y entenderlo por dentro fue clave para poder modificarlo y lograr que funcionara con nuestras propias trayectorias y simulaciones.

Este proyecto me ayudó a afianzar mis conocimientos en robótica y me motivó a seguir explorando áreas como control, automatización y simulación, que antes me parecían demasiado complicadas. Ahora me siento más preparado para enfrentar proyectos más grandes y con más confianza para resolver problemas técnicos.

