\section{Moreno Ana}
Participar en este proyecto me sacó completamente de mi zona de confort. Yo tenía más experiencia en diseño mecánico que en programación, así que cuando llegó el momento de usar MATLAB para desarrollar la cinemática, me sentí algo perdida. Pero con ayuda del equipo y revisando el código descargado del repositorio, fui entendiendo cada parte y aprendí mucho sobre cómo se implementan los modelos matemáticos de un robot.

Me gustó especialmente ver cómo la teoría que estudiamos en clase tomaba forma en algo visual y tangible, como ver el robot moverse en Gazebo o probar diferentes trayectorias. Otro aspecto que disfruté fue trabajar con los sensores simulados y pensar en cómo los incorporaríamos en un caso real. Este proyecto me dejó con muchas ganas de seguir aprendiendo sobre integración de sistemas y simulación robótica, y me demostró que incluso cuando las cosas no salen como uno quiere, siempre se aprende algo nuevo.