\chapter{Introducción} \label{chap:introduccion}

El presente trabajo tiene como objetivo documentar y analizar el desarrollo de un proyecto robótico realizado durante el semestre, el cual integra los principales conceptos estudiados en la asignatura de Robótica. Para este proyecto, se seleccionó un modelo de robot descargado directamente desde una plataforma en línea, el cual ya contaba con un diseño de ensamblado estructural completo. Esta elección permitió enfocar el trabajo en el análisis cinemático y en la aplicación de modelos matemáticos y conceptuales clave.

Una vez definido el modelo, se procedió a identificar sus ejes de rotación y sus coordenadas espaciales, lo que permitió establecer los parámetros necesarios para construir las tablas de Denavit-Hartenberg. A partir de estas tablas, se desarrolló la cinemática directa del robot, permitiendo determinar la posición y orientación del efector final en función de las variables articulares. Posteriormente, se abordó la cinemática diferencial, lo cual facilitó el análisis de velocidades y permitió una mejor comprensión del comportamiento dinámico del sistema.

Además, se resolvió el problema de la cinemática inversa, fundamental para el control del robot en tareas específicas, ya que permite calcular los valores articulares necesarios para alcanzar una determinada posición en el espacio. Todos estos análisis y cálculos se realizaron en concordancia con los contenidos abordados a lo largo del semestre, consolidando así los conocimientos teóricos mediante su aplicación práctica.

Este informe detalla cada una de estas etapas, presentando los fundamentos teóricos, los métodos aplicados y los resultados obtenidos, con el fin de demostrar el dominio adquirido en el manejo de herramientas y conceptos fundamentales en el campo de la robótica.