\chapter{Marco Teórico} 
\label{chap:marco_teorico}

El desarrollo de sistemas robóticos se fundamenta principalmente en el conocimiento de la cinemática y la dinámica, los cuales permiten modelar, estructurar y controlar el movimiento de los manipuladores. Estos conceptos son esenciales para comprender el comportamiento físico y matemático de los robots, y para diseñar sistemas de control eficientes.

En este apartado se abordan los fundamentos teóricos necesarios para el análisis, control y simulación de manipuladores robóticos. Se destacan tres pilares fundamentales: la cinemática, encargada de describir el movimiento de los eslabones y articulaciones sin considerar las fuerzas; la dinámica, que analiza las fuerzas y torques que intervienen en dicho movimiento; y el uso de ROS (Robot Operating System), una herramienta de software ampliamente utilizada para la simulación y control de sistemas robóticos complejos.

Estos tres elementos conforman la base sobre la cual se desarrollan soluciones precisas y funcionales en el campo de la robótica moderna, facilitando tanto la experimentación virtual como la implementación en entornos reales.

