\section{Proceso de Cinemática} \label{sec:proceso_cinematica}

En esta sección se describe el proceso para calcular las diferentes etapas de la cinemática del robot desarrollado, el cual fue descargado inicialmente de internet y adaptado para nuestro uso. Todo el código está disponible en el repositorio de GitHub, por lo que aquí se presentarán solo las partes más relevantes para la comprensión del trabajo.

\subsection{Cinemática Directa}

La cinemática directa se encarga de obtener la posición y orientación del efector final del robot dada una configuración específica de sus articulaciones.

Una parte fundamental del código consiste en crear la matriz homogénea de referencia inicial, que representa la orientación y posición del efector al inicio del movimiento. Para ello, se utiliza la función \texttt{euler2rotMat} que convierte ángulos de Euler a una matriz de rotación:

\begin{matlabcode}{matlab}
	R_inicial = euler2rotMat(orientacion_inicial, secuencia);
	Matriz_cero = [0 0 0];
	A0 = [R_inicial posicion_inicial;
	Matriz_cero 1];
\end{matlabcode}

Aquí, \texttt{A0} representa la matriz homogénea inicial, donde \texttt{posicion_inicial} es un vector con la posición en el espacio y \texttt{R_inicial} la orientación asociada.

Posteriormente, se leen los parámetros de Denavit-Hartenberg desde un archivo CSV para configurar el robot y sus límites de movimiento:

\begin{matlabcode}{matlab}
	dh = readtable('datos\tabla_DH\robotFinal.csv');
	robot = crear_robot(dh, A0);
\end{matlabcode}

Con el robot definido, se genera una trayectoria periódica para las articulaciones, con un período de 2 segundos:

\begin{matlabcode}{matlab}
	periodo = 2;    % Periodo del movimiento cíclico
	[q, dq, ddq] = trayectoria_q(robot, t, periodo);
\end{matlabcode}

Esto permite que las articulaciones realicen un movimiento repetitivo, facilitando el análisis del comportamiento dinámico y la validación de la cinemática.

Los resultados y gráficos relacionados se pueden consultar en el capítulo \autoref{chap:resultados}.

\subsection{Cinemática Diferencial}

La cinemática diferencial permite relacionar las velocidades articulares con la velocidad lineal y angular del efector final. En el código, se calcula el Jacobiano del robot para determinar estas relaciones.

Una parte clave del código es la función que calcula el Jacobiano y lo usa para obtener las velocidades del efector a partir de las velocidades articulares:

\begin{matlabcode}{matlab}
	J = calcular_jacobiano(robot, q);
	v_efector = J * dq;
\end{matlabcode}

Esto es fundamental para control y seguimiento de trayectoria, así como para análisis de singularidades y manipulabilidad.

Los detalles de implementación y resultados gráficos se encuentran en el capítulo \autoref{chap:resultados}.

\subsection{Cinemática Inversa}

Para la cinemática inversa, se busca calcular los valores de las articulaciones que lleven al efector a una posición y orientación deseadas. El método implementado utiliza un algoritmo iterativo con control de convergencia.

La función principal recibe como entradas el robot, la posición deseada, la tolerancia, el máximo número de iteraciones, el parámetro de ajuste y el número de muestras:

\begin{matlabcode}{matlab}
	function [q_sol, p_sol] = cinematica_inv(r, p_des, tol, max_iter, alpha, numMuestras)
\end{matlabcode}

Este algoritmo itera ajustando las articulaciones hasta que la posición obtenida esté dentro de la tolerancia requerida o se alcance el número máximo de iteraciones.

Los resultados de este procedimiento, así como las gráficas de convergencia y trayectorias, están documentados en el capítulo \autoref{chap:resultados}.
